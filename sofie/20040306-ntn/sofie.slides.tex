\documentclass[a4paper,landscape]{slides}

\usepackage[dvips]{color}
\usepackage{times,mathptm,isolatin1,graphicx,amsmath}
\usepackage{amscd,verbatim}
\usepackage[danish,english]{babel}

\long\def\handout#1{}
\long\def\slides#1{#1}

\newcounter{sentence}
\def\sent{(\refstepcounter{sentence}\arabic{sentence})\ \ }
\setlength{\textwidth}{20cm}
\def\TITLE#1{\textbf{\colorbox{blue}{\textcolor{white}{\large #1}}}}
\def\title#1{\textbf{\colorbox{blue}{\textcolor{white}{#1}}}}
\def\strong#1{\textbf{\textcolor{red}{#1}}}
\def\high#1{\textbf{\textcolor{blue}{#1}}}
\def\into{\longrightarrow}
\def\implies{\rightarrow}
\def\slidegraphics#1{\includegraphics[scale=1.8]{#1}}
\def\slidegraphicsx#1#2{\includegraphics[#1]{#2}}
\def\invert#1{\textbf{\colorbox{cyan}{#1}}}
\def\flash#1{\textbf{\colorbox{yellow}{#1}}}
\newenvironment{itemlist}
    {\begin{list}{$\bullet$}{\setlength{\itemsep}{0mm}%
		\setlength{\parsep}{0mm}{}}}
	{\end{list}}
\newenvironment{zlide}
	{\begin{slide}\setlength{\parskip}{4mm}}{\vfill\end{slide}}

\begin{document}
\def\map#1#2#3{#1$\begin{CD} @>{\text{\small #2}}>> \end{CD}$#3}
\def\imap#1#2#3{#1$\begin{CD} @<{\text{\small #2}}<< \end{CD}$#3}
\def\vcenterbox#1{$\begin{array}[c]{l}\text{#1}\end{array}$}


\newcounter{mypart}
\newcounter{myslide}

\def\sor{\texttt{|}}

\def\derive#1#2#3{
	\begin{CD}
		\begin{array}[c]{l}#1\end{array}
		@>{\text{#2}}>> 
		\begin{array}[c]{l}#3\end{array}
	\end{CD}}

\def\ul#1{\underline{\smash{#1}}}
    \newcommand{\typ}[1]{\emph{#1}}
    \newcommand{\var}[1]{\emph{#1}}
    \newcommand{\val}[1]{\emph{#1}}
    \newcommand{\abs}[1]{\ensuremath{|#1|}}
    \newcommand{\ab}[1]{\ensuremath{|\text{#1}|}}

    \newcommand{\isaop}{\text{isa}}
    \newcommand{\thisop}{\text{this}}
    \newcommand{\depop}{\text{dep}}
    \newcommand{\repairop}{\text{repair}}
    \newcommand{\landop}{\text{land}}
    \newcommand{\lexop}{\text{lex}}
    \newcommand{\leftop}{\text{left}}
    \newcommand{\rightop}{\text{right}}
    \newcommand{\islandop}{\text{island}}
    \newcommand{\andop}{\text{and}}
    \newcommand{\orop}{\text{or}}
    \newcommand{\notop}{\text{not}}
    \newcommand{\semop}{\text{sem}}
    \newcommand{\govop}{\text{gov}}
    \newcommand{\distop}{\text{dist}}
    \newcommand{\lsiteop}{\text{lsite}}
    \newcommand{\boldf}[1]{\textbf{$\mathbf{#1}$}}
    \newcommand{\defeq}{:=}
    \newcommand{\code}[1]{{\textsf{\small#1}}}
    \newcommand{\ptitle}[1]{\multicolumn{2}{|c|}{\textbf{#1}}}
    \newcommand{\sel}[1]{\textbf{#1}}
    \newcommand{\lex}[1]{lex(#1)}
    \newcommand{\landed}[2]{land(#1$^{\text{#2}}$)}
    \newcommand{\dep}[3]{dep(#1$^{\text{#2}}_{\text{#3}}$)}
        \def\repairl#1#2#3{repair1(#1$^{\text{#2}}_{\text{#3}}$)}
            \def\repair#1#2#3#4#5#6{repair2(#1$^{\text{#2}}_{\text{#3}}$,#4$^{\text{#5}}_{\text{#6}}$)}
                \def\repairx#1#2#3#4#5#6#7#8#9{repair3(#1$^{\text{#2}}_{\text{#3}}$,#4$^{\text{#5}}_{\text{#6}}$,#7$^{\text{#8}}_{\text{#9}}$)}


\def\code#1{\texttt{#1}}
\def\derive#1#2#3{
    \begin{CD}
        \begin{array}[c]{l}#1\end{array}
        @>{\text{#2}}>> 
        \begin{array}[c]{l}#3\end{array}
    \end{CD}}

\begin{zlide}
	\begin{center}
		{\large\bfseries\color{blue}The DDT annotations \\ 
		in the 
			parallel Sofie treebank}
        \bigskip

		Matthias Trautner Kromann 

        Lovik meeting of the Nordic Treebank Network \break
		March 6, 2004
	
		\includegraphics[scale=0.5]{figs/cmol-logo.eps}

		Center for Computational Modelling of Language\break
        Department of Computational Linguistics \break
		Copenhagen Business School\break
        http://www.id.cbs.dk/$\sim$mtk\medskip
    \end{center}
\end{zlide}

\begin{zlide}
	\title{1. Construction of DDT annotated Danish part of Sofie
	treebank}

	The DDT-annotated part of the parallel Sofie treebank consists of
	\high{100 sentences} annotated according to the \high{DDT
	annotation scheme} (http://www.id.cbs.dk/$\sim$mtk/treebank). 

	The treebank has been constructed by \high{manual annotation}, at
	a \high{rate of 1.000--1.500 words per working day}. The
	PAROLE tag (the ``msd'' feature) was created with a \high{primitive
	tagger and manual post-editing} (I could have used a Brill tagger
	for this task, but I didn't have it on my computer, so I did the
	annotation with a simple program instead).
\end{zlide}

\begin{zlide}
	\title{2. Choice of category and edge labels in DDT}

	\begin{itemlist}
		\item \high{linguistic theory}: dependency-based
		\item \high{category labels}: encode word class and
			morphological features (number, case, gender, verbal
			inflection, etc.), according to the standards from the EU
			funded PAROLE project.
		\item \high{edge labels}: complement and adjunct labels
			are DDT-specific, but most labels are also found under
			different names in other treebank projects. 
	\end{itemlist}
	I don't know whether it is possible to \strong{standardize category and
	edge labels}, but it might be worth a try.
\end{zlide}

\begin{zlide}
	\title{3. Comparing DDT analyses with other analyses in the
		Sofie treebank}
	\nopagebreak
	{\small
	\begin{tabular}{p{50mm}|p{38mm}|p{38mm}|p{38mm}|p{38mm}}
			& \textbf{Bick} & \textbf{Kromann} & \textbf{Nivre} &
			\textbf{Volk} \\ \hline
		\textbf{framework} &
			phrase &
			dependency &
			dependency &
			phrase \\ \hline
		\textbf{discont.} &
			yes &
			yes &
			no &
			yes \\ \hline
		\textbf{fillers/coreference} &
			no &
			yes &
			no &
			no \\ \hline
		\textbf{NP/DP anal.} 
			& NP
			& DP
			& NP
			& parataxis \\ \hline
		\textbf{VP compounds} &
			verb cluster &
			nested comps & 
			nested comps &
			nested comps \\ \hline
		\textbf{coordination} & 
			parataxis &
			modify 1st &
			modify 1st &
			parataxis \\ \hline
		\textbf{punctuation} &
			no &
			yes &
			yes &
			no \\ \hline
		\textbf{apposition head} &
			1st N &
			1st N &
			2nd N &
			parataxis \\ \hline
		\textbf{it-cleft: it ... \emph{that}} &
			formal/real subject &
			it-complement &
			it-apposition &
			direct object \\ \hline

		\textbf{comp/adj: opst� \emph{af ...}} &
			sentential adjunct &
			complement & 
			adjunct & 
			adjunct \\ \hline
		\textbf{discont. AP: s� lang en vej \emph{som}} &
			s�-adjunct &
			s�-adjunct &
			V-adjunct &
			N-adjunct \\ \hline
		\textbf{wh-clause: VP1 \emph{wh} VP2} &
			adjunct within VP2 &
			wh+VP2 head, VP2 filler&
			adjunct within VP2 &
			adjunct within VP2 \\ \hline
	\end{tabular}}
\end{zlide}

\begin{zlide}
	\title{4. Strengths, weaknesses, threats and opportunities}

	Weaknesses and threats:
	\begin{itemlist}
		\item \strong{Important differences} in syntactic analysis.
		\item A \strong{common analysis scheme} is unrealistic. 
		\item \strong{Translation between frameworks} is likely
			to be difficult. 
	\end{itemlist}
	Strengths and opportunities:
	\begin{itemlist}
		\item TIGER XML has made it easier to \strong{compare analyses
			and exchange ideas}.
		\item \strong{Common edge and
			category labels} is worth a try: similar
			labels should get a standardized name, remaining labels 
			should be added as theory-specific.
		\item \strong{Standardized extensions of TIGER
			XML} for spoken language, dialogue, and discourse are
			worth a try.
	\end{itemlist}
\end{zlide}

\begin{zlide}
	\title{5. Proposed tasks in "Parallel treebanks"}

	\begin{itemlist}
		\item Compare and document existing \high{category
			and edge labels}.
		\item Create an inclusive set of \high{common
			category labels}. 
		\item Create an inclusive set of \high{common edge
			labels}. 
	\end{itemlist}
	The following task is delegated to "Tools and Resources":
	\begin{itemlist}
		\item Standardize \high{extensions to TIGER XML} for spoken
			language, dialogue, and discourse.
	\end{itemlist}
\end{zlide}
\end{document}
