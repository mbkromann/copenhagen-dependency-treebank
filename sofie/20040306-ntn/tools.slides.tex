\documentclass[a4paper,landscape]{slides}

\usepackage[dvips]{color}
\usepackage{times,mathptm,isolatin1,graphicx,amsmath}
\usepackage{amscd,verbatim}
\usepackage[danish,english]{babel}

\long\def\handout#1{}
\long\def\slides#1{#1}

\newcounter{sentence}
\def\sent{(\refstepcounter{sentence}\arabic{sentence})\ \ }
\setlength{\textwidth}{20cm}
\def\TITLE#1{\textbf{\colorbox{blue}{\textcolor{white}{\large #1}}}}
\def\title#1{\textbf{\colorbox{blue}{\textcolor{white}{#1}}}}
\def\strong#1{\textbf{\textcolor{red}{#1}}}
\def\high#1{\textbf{\textcolor{blue}{#1}}}
\def\into{\longrightarrow}
\def\implies{\rightarrow}
\def\slidegraphics#1{\includegraphics[scale=1.8]{#1}}
\def\slidegraphicsx#1#2{\includegraphics[#1]{#2}}
\def\invert#1{\textbf{\colorbox{cyan}{#1}}}
\def\flash#1{\textbf{\colorbox{yellow}{#1}}}
\newenvironment{itemlist}
    {\begin{list}{$\bullet$}{\setlength{\itemsep}{0mm}%
		\setlength{\parsep}{0mm}{}}}
	{\end{list}}
\newenvironment{zlide}
	{\begin{slide}\setlength{\parskip}{4mm}}{\vfill\end{slide}}

\begin{document}
\def\map#1#2#3{#1$\begin{CD} @>{\text{\small #2}}>> \end{CD}$#3}
\def\imap#1#2#3{#1$\begin{CD} @<{\text{\small #2}}<< \end{CD}$#3}
\def\vcenterbox#1{$\begin{array}[c]{l}\text{#1}\end{array}$}


\newcounter{mypart}
\newcounter{myslide}

\def\sor{\texttt{|}}

\def\derive#1#2#3{
	\begin{CD}
		\begin{array}[c]{l}#1\end{array}
		@>{\text{#2}}>> 
		\begin{array}[c]{l}#3\end{array}
	\end{CD}}

\def\ul#1{\underline{\smash{#1}}}
    \newcommand{\typ}[1]{\emph{#1}}
    \newcommand{\var}[1]{\emph{#1}}
    \newcommand{\val}[1]{\emph{#1}}
    \newcommand{\abs}[1]{\ensuremath{|#1|}}
    \newcommand{\ab}[1]{\ensuremath{|\text{#1}|}}

    \newcommand{\isaop}{\text{isa}}
    \newcommand{\thisop}{\text{this}}
    \newcommand{\depop}{\text{dep}}
    \newcommand{\repairop}{\text{repair}}
    \newcommand{\landop}{\text{land}}
    \newcommand{\lexop}{\text{lex}}
    \newcommand{\leftop}{\text{left}}
    \newcommand{\rightop}{\text{right}}
    \newcommand{\islandop}{\text{island}}
    \newcommand{\andop}{\text{and}}
    \newcommand{\orop}{\text{or}}
    \newcommand{\notop}{\text{not}}
    \newcommand{\semop}{\text{sem}}
    \newcommand{\govop}{\text{gov}}
    \newcommand{\distop}{\text{dist}}
    \newcommand{\lsiteop}{\text{lsite}}
    \newcommand{\boldf}[1]{\textbf{$\mathbf{#1}$}}
    \newcommand{\defeq}{:=}
    \newcommand{\code}[1]{{\textsf{\small#1}}}
    \newcommand{\ptitle}[1]{\multicolumn{2}{|c|}{\textbf{#1}}}
    \newcommand{\sel}[1]{\textbf{#1}}
    \newcommand{\lex}[1]{lex(#1)}
    \newcommand{\landed}[2]{land(#1$^{\text{#2}}$)}
    \newcommand{\dep}[3]{dep(#1$^{\text{#2}}_{\text{#3}}$)}
        \def\repairl#1#2#3{repair1(#1$^{\text{#2}}_{\text{#3}}$)}
            \def\repair#1#2#3#4#5#6{repair2(#1$^{\text{#2}}_{\text{#3}}$,#4$^{\text{#5}}_{\text{#6}}$)}
                \def\repairx#1#2#3#4#5#6#7#8#9{repair3(#1$^{\text{#2}}_{\text{#3}}$,#4$^{\text{#5}}_{\text{#6}}$,#7$^{\text{#8}}_{\text{#9}}$)}


\def\code#1{\texttt{#1}}
\def\derive#1#2#3{
    \begin{CD}
        \begin{array}[c]{l}#1\end{array}
        @>{\text{#2}}>> 
        \begin{array}[c]{l}#3\end{array}
    \end{CD}}

\begin{zlide}
	\title{NTN Tools and Resources session. Lovik, March 6, 2004}

	\begin{tabular}{p{20mm}p{210mm}}
		 & \strong{Experiences with TIGER XML} \\ 
		09:00	& \high{Eckhard Bick}: TIGERizing Arboretum --- feeding VISL
				to the beast. \\
		09:25 & \high{Joakim Nivre}: Converting MALT XML to TIGER XML. \\
		09:50	 	& \high{Jens Allwood}: Opinion on spoken language +
			G�teborg tools and resources.  \\
		10:15 & \high{Matthias T. Kromann}: Converting DDT to TIGER
XML.\\ \hline

		10.30 & Coffee break. \\ \hline

		 & \strong{From syntax to discourse, dialogue, and
			speech} \\
		11:00 & \high{Martin Volk}: The ANNOTATE tree editor. \\
		11:25 & \high{Sofia Gustafson-Capkova}: Opinion on discourse
			treebanks.\\ \hline
		11:40 & \strong{General discussion.}\\ \hline
		12:30 & Lunch.
	\end{tabular}
\end{zlide}

\begin{zlide}
	\begin{center}
		{\large\bfseries\color{blue}Converting the Danish Dependency
		Treebank \\ to TIGER XML}
        \bigskip

		Matthias Trautner Kromann 

        Lovik meeting of the Nordic Treebank Network \break
		March 6, 2004
	
		\includegraphics[scale=0.5]{figs/cmol-logo.eps}

		Center for Computational Modelling of Language\break
        Department of Computational Linguistics \break
		Copenhagen Business School\break
        http://www.id.cbs.dk/$\sim$mtk\medskip
    \end{center}
\end{zlide}

\begin{zlide}
	\title{1. Conversion of DDT to TIGER XML}

	The conversion was done by implementing a \high{TIGER XML export
	filter} in the annotation program DTAG, so that any treebank
	stored in native DTAG format can be exported to TIGER XML.

	The export filter took approximately \high{8 hours to program}.
	The time-consuming bits were: 
	\begin{itemlist}
		\item \strong{XML header}: making an inventory of all
			edge types and feature values in the text (for features
			with less than 500 values).
		\item \strong{sentence segmentation}: segmenting the text into
			sentence-sized graphs. 
		\item \strong{primary and secondary edges}: distinguishing
			between primary and secondary edges
	\end{itemlist}
\end{zlide}


\begin{zlide}
	\title{2. Conversion rules}
	
	\begin{itemlist}
		\item \high{phrase-structure graphs}: words in DTAG are
			converted into two TIGER nodes: a terminal and a
			non-terminal node, linked with a ``\texttt{--}'' primary
			edge.

		\item \high{features}: any variable-value pair associated with
			a word in the DTAG format is output as a feature of the
			corresponding terminal and non-terminal node in TIGER XML;
			words are stored in non-terminal feature ``word''.

		\item \high{sentence segmentation}: the text is segmented into
			maximal, connected primary trees, and these are output as
			sentences/graphs in TIGER XML.

		\item \high{primary/secondary edge distinction}: DTAG allows
			the user to control whether edges are printed on top of
			the words, or below them; during export, edges below the
			words are analyzed as secondary, all other edges are
			analyzed as primary.
	\end{itemlist}
\end{zlide}

\begin{zlide}
	\title{3. Example of graph translated from DTAG to TIGER XML}

	\[
		\slidegraphics{figs/coref.eps}
	\]
	\[
		\slidegraphicsx{width=160mm,height=55mm}{figs/coref-tiger.eps}
	\]
\end{zlide}

\begin{zlide}
	\title{4. Conversion successes, failures, and potential problems}

	\begin{itemlist}
		\item \high{Secondary edges}: Coreference and filler edges 
			are preserved in translation. 
		\item \high{Cross-sentential edges}: Edges between different
			sentences are lost in translation.
		\item \high{Dependency representation}: Dependency graphs
			cannot be represented as words connected with edges, they
			must be represented as \strong{discontinuous
			phrase-structure graphs}, thereby doubling the number of
			nodes and edges in the graph.
		\item \high{Well-formedness in TIGER}: The exported TIGER XML
			file is not guaranteed to contain only well-formed TIGER
			graphs (but the Danish Dependency Treebank led to few
			errors).
	\end{itemlist}
\end{zlide}

\begin{zlide}
	\title{5. Comparison between DTAG format and TIGER XML}

	\nopagebreak
	\begin{tabular}{p{50mm}|p{80mm}|p{80mm}}
		& \textbf{TIGER XML} & \textbf{DTAG} \\ \hline

		\high{segmentation} &
			sentences with words &
			texts with words \\ \hline

		\high{edges crossing sentences} &
			ill-formed/ignored &
			unproblematic \\ \hline

		\high{node types} &
			terminal/non-ter\-mi\-nal: non-terminals must
			do\-mi\-na\-te
			some terminal &
			no distinction \\ \hline

		\high{edge types} &
			primary/secondary & 
			no distinction \\ \hline

		\high{graph types} & 
			primary edges must form a tree; 
			secondary edges unrestricted  & 
			unrestricted: general digraphs \\  \hline

		\high{XML scheme} & 
			restricted: document must follow strict DTD & 
			unrestricted: recognize $\langle$W$\rangle$
			tags, treat everything else as comments \\ \hline

		\high{XML compliant} &
			yes &
			optional \\ \hline
	\end{tabular}

\end{zlide}



\begin{zlide}
	\title{6. Strengths and weaknesses of TIGER XML}
	
	Strengths:
	\begin{itemlist}
		\item \high{simplicity}: format is simple and easy to
			implement.
		\item \high{phrase/dependency support}: works with
			both phrase-structure and dependency.
		\item \high{discontinuity}: can handle discontinuities.
		\item \high{secondary edges}: supports secondary dependencies. 
	\end{itemlist}
	Weaknesses:
	\begin{itemlist}
		\item \high{sentence-based}: no support for edges between
			sentences, hence no support for discourse annotation.
		\item \high{text-based}: no support for linking nodes
			to external sound/video segments.
		\item \high{visualization}: not geared towards text-sized segments. 
	\end{itemlist}
\end{zlide}

\begin{zlide}
	\title{7. Proposed work-arounds and adaptations of TIGER XML}

	\begin{itemlist}
		\item \high{cross-sentential links}: encode cross-sentential
			links with secondary edges, ignoring warnings from
			TIGERRegistry; ask TIGER team to implement cross-sentential
			links.
		\item \high{dependency graphs}: facilitate exchange of
			dependency treebanks by standardizing the way 
			dependency graphs are represented (eg, use the edge label
			"\texttt{--}", "HD", or "HEAD" to indicate the head word in the
			phrase).
		\item \high{separate source and annotation}: allow separation
			of source (text/sound/video) and annotation by
			introducing additional "segment" attribute that indicates
			(discontinuous) segment in source file; standardize
			features, eg: \texttt{<}t
			word="This" speaker="A" source="/opt/ddt/talk12.mp3"
			segment="13-35,77-92"\texttt{>}. 
	\end{itemlist}
\end{zlide}

\begin{zlide}
	\title{8. Proposed tasks in "Tools and Resources"}

	\begin{itemlist}
		\item Standardize \strong{representation of cross-sentential
			links with secondary edges}.
		\item Standardize \strong{dependency representation}
			in TIGER XML.
		\item Standardize \strong{speech-related features in TIGER XML}: 
			\begin{itemlist}
				\item \high{source}: source file for text/sound/video 
				\item \high{segment}: list of segments within the
					source file covered by the annotation.
				\item \high{speaker}: speaker associated with
					segment.
			\end{itemlist}
	\end{itemlist}
\end{zlide}



\end{document}
