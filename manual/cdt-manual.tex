\documentclass[11pt]{report}

\usepackage{hyperref}
\usepackage{fontspec}
\usepackage{xunicode}
\usepackage{xltxtra}
\usepackage{marginnote}
\usepackage{ifthen}
\usepackage{makeidx}
\usepackage{a4}
\usepackage{sectsty}
\allsectionsfont{\sffamily}

\hypersetup{
	pdfborder = {0 0 0}
}


\usepackage{multicol}
\makeatletter
\renewenvironment{theindex}
  {\if@twocolumn
      \@restonecolfalse
   \else
      \@restonecoltrue
   \fi
   \newpage
   \setlength{\columnseprule}{0pt}
   \setlength{\columnsep}{35pt}
   \begin{multicols}{3}[\chapter{\indexname}]
   \markboth{\MakeUppercase\indexname}%
            {\MakeUppercase\indexname}%
%   \thispagestyle{plain}  % I don't want the first page of the index
%   to be plain
   \setlength{\parindent}{0pt}
   \setlength{\parskip}{0pt plus 0.3pt}
   \relax
   \let\item\@idxitem}%
  {\end{multicols}\if@restonecol\onecolumn\else\clearpage\fi}
\makeatother

\makeindex
%\setmainfont[Mapping=tex-text]{Bitstream Charter}

% Packages
%\usepackage{pstricks}  % since the dash is rendered by pstricks!
%\usepackage[postscript]{ucs}
%\usepackage[utf8x]{inputenc} 
%\usepackage[T1]{fontenc}
\usepackage{graphicx}

% Macros
\newenvironment{relation}{\medskip\hangindent=0mm}{\medskip}
\newenvironment{examples}{\medskip\par\begin{center}}{\end{center}}
\newenvironment{ldescription}{}{\smallskip\par\hangindent=5mm}
\newenvironment{overview}[1]{\begin{figure}[!ht]\def\mycaption{#1}\centering\begin{tabular}{|p{120mm}|}\hline}{\\ \hline\end{tabular}\caption{The relations matching \rel{\small\mycaption}.}\end{figure}}

\def\rel#1{\textsf{#1\index{#1}}}
\def\reldef#1#2{\label{#1}#2}
\def\relref#1#2{\hyperref[#1]{#2}}
\def\relname#1#2#3{\par\noindent\llap{\smash{\parbox[t]{40mm}
	{\raggedleft\textbf{#1}#2#3}\ \ }}}
\def\isa#1{\\\small isa #1}
\def\lineno#1{\\\small [#1]}

\def\sdescr#1{\textit{#1}.}
\def\sdescrx#1#2{\textit{#1} {\footnotesize (#2)}.}
\def\xlong#1{long: #1}
\def\deprecated#1{deprecated #1}
\def\subtypes#1{\par\footnotesize\hangindent=5mm\noindent Subtypes: #1. }
\def\related#1{\par\footnotesize\hangindent=5mm\noindent Related types: #1. }
\def\confusions#1#2{\par\footnotesize\hangindent=5mm\noindent Confusion$_{#1}$: #2. }
\def\confuse#1#2{#2$_{#1\%}$ }
\def\tparents#1{}
\def\cmdsummary#1#2#3{\small\phantom{.}#1#2: #3\newline}
\def\overviewfile#1{\input{#1}}

\def\mytab{\hspace{5mm}}
\def\exfig#1{\quad\includegraphics{#1}\quad\allowbreak}

\textheight=240mm
\textwidth=140mm
\topmargin=0mm
\headsep=0mm
\headheight=0mm
\oddsidemargin=10mm
\evensidemargin=10mm


\begin{document}
	\title{\sffamily\bfseries The inventory of linguistic relations used in the 
		Copenhagen Dependency Treebanks}
	\author{Matthias Buch-Kromann \and 
		Morten Gylling-Jørgensen \and
		Lotte Jelsbech Knudsen \and
		Iørn Korzen \and
		Henrik Høeg Müller \\[10pt]
		%Center for Research in Translation and Translation Technology \\
		Dept.\ of International Language Studies and Computational
		Linguistics \\
		Copenhagen Business School}
	\maketitle

	\begin{abstract}
		This manual describes the inventory of linguistic relations used in the
		Copenhagen Dependency Treebanks, a set of parallel treebanks
		for Danish, English, German, Italian, and Spanish annotated
		with respect to syntax, morphology, discourse, coreference,
		and translational equivalence. The manual is generated
		automatically from the CDT project's online relation
		spreadsheet.\footnote{\url{http://spreadsheets.google.com/ccc?key=0ArjTKYTQS1lWcnNUWGJrX3lZTkxDc3QxYmlqWlRXQ1E&hl=en}}
	\end{abstract}

	\tableofcontents

	\newpage

	\chapter{Introduction}

	This manual describes the relations used in the Copenhagen
	Dependency Treebanks. The relations are ordered in a hierarchy,
	where each relation may have zero or more immediate super types,
	and zero or more immediate subtypes. The relations are presented
	in detail in the following chapters, grouped by linguistic level
	and general relation type. Every time a relation is introduced,
	its name is written in the left margin, with an indication of its
	immediate super types and the row in the online CDT spreadsheet
	in which the relation was defined. An example is shown
	below.\medskip

	\relname{\rel{relation}}{\isa{\rel{super}}}{\lineno{12}}%
	The notation in the left margin indicates that we now describe 
	the relation \rel{relation}; it has immediate super type
	\rel{super} and is defined in row $12$ in the spreadsheet. When
	describing a relation, we also lists its other properties, if
	relevant, including its:
	\begin{itemize}
		\item \textit{long name}: we use short names in the annotation
			for brevity, but long names are sometimes more descriptive, so
			we provide these as an alias for the short relation name;
		\item \textit{deprecated names}: when renaming relations, 
			the old name is listed as a deprecated name for backwards
			compatibility, but it should be avoided in future
			annotation;
		\item \textit{immediate subtypes}: the relation names that 
			have been specified as the immediate subtypes of the
			relation;
		\item \textit{related types}: lists the relations that 
			are closely related to this relation, in some way or
			another, and which you might want to consult for
			clarification or additional information;
		\item \textit{examples}: small annotated text examples that
			illustrate how the relation is used;
	\end{itemize}
	In PDF versions of this document, relation names are clickable so
	that you can navigate through the relation hierarchy by clicking
	on the relation names.
	
	\input{cdt-rels}

	\appendix

	\chapter{Overview tables}

	The tables in this section lists all the relations in the
	Copenhagen Dependency Treebanks, repeated from the preceding
	sections.

	\renewenvironment{overview}[1]{\begin{center}\def\mycaption{#1}\begin{tabular}{|p{120mm}|}\hline}{\\ \hline\end{tabular}\nopagebreak\medskip\nopagebreak\par\nopagebreak{The relations matching \rel{\small\mycaption}.}\end{center}\bigskip}

	\input{build/cdt-rels-overviews.tex}

	\printindex

\end{document}
